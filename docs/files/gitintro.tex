%!TEX root = ../main.tex

\begin{frame}[t]\frametitle{Introducing Git}
    Git is a distributional version control system most notably used
    by Github, the web-based hosting service for software development projects.

    We will use Git to demonstrate version control systems. For a basic
    introduction to git, we recommend \citet{ChaconHamano2009}, which
    can be accessed \href{http://git-scm.com/book}{here}. We
    borrow heavily from this book here.
\end{frame}

\begin{frame}[t]\frametitle{Introducing Git}
    First consider a git server, which is nothing but a computer
    with the following file structure.

    \begin{center}
        \tikzstyle{file}=[draw, rectangle, text centered, text width=2.5cm,
            fill=black!20, rounded corners]

        \begin{tikzpicture}

            \begin{scope}[local bounding box=bb]
                \node (server) {Central Server};
                \node [file, below of=server] (folder) {MyProject/};
                \node [file,below of=folder,xshift=2cm] (a) {File a};
                \node [file,below of=a] (b) {File b};
                \node [file,below of=b] (c) {File c};
                \node [file,below of=c,xshift=-2cm] (git) {.git/};
                \node [below of=git,xshift=0.3cm] (content) {...};

                \draw (folder) -- (0,-3);
                \draw (0,-3) -- (a.west);
                \draw (0,-3) -- (b.west);
                \draw (0,-3) -- (c.west);
                \draw (0,-3) -- (git.north);
                \draw (git.south) -- (content.west);
            \end{scope}

            \begin{pgfonlayer}{background}
                \node [fill=blue!20,fit=(bb)] {};
            \end{pgfonlayer}

            \draw [decorate,decoration={brace,amplitude=10pt}] (4,0) -- (4,-6) node [midway,xshift=1.5cm] (hash) {24b9da655};
            \node [right of=git,xshift=6cm,yshift=-1cm] (gitlab) {Git Storage};
            \node [below of=hash,xshift=1cm] (hashlab) {Commit Hash (version \#)};

            \draw [<-] (hash) -- (hashlab);
            \draw [<-] (git) -- (gitlab);
        \end{tikzpicture}
    \end{center}
\end{frame}

\begin{frame}[t]\frametitle{Introducing Git}
    When Pam clones this repository to her computer, she sees:
    \begin{center}
        \tikzstyle{file}=[draw, rectangle, text centered, text width=2.5cm,
            fill=black!20, rounded corners]

        \begin{tikzpicture}

            \begin{scope}[local bounding box=bb]
                \node (server) {Pam's Computer};
                \node [file, below of=server] (folder) {MyProject/};
                \node [file,below of=folder,xshift=2cm] (a) {File a};
                \node [file,below of=a] (b) {File b};
                \node [file,below of=b] (c) {File c};
                \node [file,below of=c,xshift=-2cm] (git) {.git/};
                \node [below of=git,xshift=0.3cm] (content) {...};

                \draw (folder) -- (0,-3);
                \draw (0,-3) -- (a.west);
                \draw (0,-3) -- (b.west);
                \draw (0,-3) -- (c.west);
                \draw (0,-3) -- (git.north);
                \draw (git.south) -- (content.west);
            \end{scope}

            \begin{pgfonlayer}{background}
                \node [fill=blue!20,fit=(bb)] {};
            \end{pgfonlayer}

            % \draw [decorate,decoration={brace,amplitude=10pt}] (4,0) -- (4,-6) node [midway,xshift=1.5cm] (hash) {24b9da655};
            % \node [right of=git,xshift=6cm,yshift=-1cm] (gitlab) {Git Storage};
            % \node [below of=hash,xshift=1cm] (hashlab) {Commit Hash (version \#)};

            % \draw [<-] (hash) -- (hashlab);
            % \draw [<-] (git) -- (gitlab);
        \end{tikzpicture}
    \end{center}
\end{frame}

\begin{frame}[t]\frametitle{Introducing Git}
    When Pam changes ``File b'', she merely changes her ``working directory''.
    Git will recognize the change, but won't record it.
    \begin{center}
        \tikzstyle{file}=[draw, rectangle, text centered, text width=2.5cm,
            fill=black!20, rounded corners]

        \begin{tikzpicture}

            \begin{scope}[local bounding box=bb]
                \node (server) {Pam's Computer};
                \node [file, below of=server] (folder) {MyProject/};
                \node [file,below of=folder,xshift=2cm] (a) {File a};
                \node [file,below of=a,fill=red!70] (b) {File b};
                \node [file,below of=b] (c) {File c};
                \node [file,below of=c,xshift=-2cm] (git) {.git/};
                \node [below of=git,xshift=0.3cm] (content) {...};

                \draw (folder) -- (0,-3);
                \draw (0,-3) -- (a.west);
                \draw (0,-3) -- (b.west);
                \draw (0,-3) -- (c.west);
                \draw (0,-3) -- (git.north);
                \draw (git.south) -- (content.west);
            \end{scope}

            \begin{pgfonlayer}{background}
                \node [fill=blue!20,fit=(bb)] {};
            \end{pgfonlayer}

            % \draw [decorate,decoration={brace,amplitude=10pt}] (4,0) -- (4,-6) node [midway,xshift=1.5cm] (hash) {24b9da655};
            % \node [right of=git,xshift=6cm,yshift=-1cm] (gitlab) {Git Storage};
            % \node [below of=hash,xshift=1cm] (hashlab) {Commit Hash (version \#)};

            % \draw [<-] (hash) -- (hashlab);
            % \draw [<-] (git) -- (gitlab);
        \end{tikzpicture}
    \end{center}
\end{frame}

\begin{frame}[c]\frametitle{Introducing Git}
    To record the change she performs two commands:
    \begin{enumerate}
        \item \inlinecode{\$ git add ``File a''}
        \item \inlinecode{\$ git commit -m ``I have changed File a.''}
    \end{enumerate}

    \vspace{0.5cm}
    The second command generates a commit and a corresponding commit 
    message. A \framebox{commit} is like a ``snapshot''; it records
    the current state of your files. Read more on commits 
    \href{http://git-scm.com/book/en/Getting-Started-Git-Basics}{here}.

\end{frame}

\begin{frame}[t]\frametitle{Introducing Git}
    Now Pam's local repository is at a future state, recorded
    as a new commit hash. The commit information is stored 
    in the git directory.
    \begin{center}
        \tikzstyle{file}=[draw, rectangle, text centered, text width=2.5cm,
            fill=black!20, rounded corners]

        \begin{tikzpicture}

            \begin{scope}[local bounding box=bb]
                \node (server) {Pam's Computer};
                \node [file, below of=server] (folder) {MyProject/};
                \node [file,below of=folder,xshift=2cm] (a) {File a};
                \node [file,below of=a] (b) {File b};
                \node [file,below of=b] (c) {File c};
                \node [file,below of=c,xshift=-2cm,fill=red!70] (git) {.git/};
                \node [below of=git,xshift=0.3cm] (content) {...};

                \draw (folder) -- (0,-3);
                \draw (0,-3) -- (a.west);
                \draw (0,-3) -- (b.west);
                \draw (0,-3) -- (c.west);
                \draw (0,-3) -- (git.north);
                \draw (git.south) -- (content.west);
            \end{scope}

            \begin{pgfonlayer}{background}
                \node [fill=blue!20,fit=(bb)] {};
            \end{pgfonlayer}

            \draw [decorate,decoration={brace,amplitude=10pt}] (4,0) -- (4,-6) node [draw,rectangle,fill=red!70,midway,xshift=1.5cm] (hash) {75b10da55};
            \node [right of=git,xshift=6cm,yshift=-1cm] (gitlab) {Git Storage};
            \node [below of=hash,xshift=1cm] (hashlab) {New Commit Hash};

            \draw [<-] (hash) -- (hashlab);
            \draw [<-] (git) -- (gitlab);
        \end{tikzpicture}
    \end{center}
\end{frame}

\begin{frame}[c,fragile]\frametitle{Introducing Git}
    Using \inlinecode{\$ git status}, we get the
    following output:

    \newbox{\mybox}
    \begin{lrbox}{\mybox}
    \begin{minipage}{\linewidth}
    \begin{lstlisting}[basicstyle=\footnotesize\ttfamily\color{white}]
$ git status
# on branch master 
# Your branch is ahead of 'origin/master' by 1 commit. 
# 
nothing to commit (working directory clean)  
    \end{lstlisting}
    \end{minipage}
    \end{lrbox}
    \colorbox{black}{\usebox{\mybox}}

    \vspace{0.5cm}
    The git directory stores the commit history:

    \vspace{0.5cm}
    ... $\rightarrow$ \colorbox{black!30}{24b9da655} $\rightarrow$ \colorbox{black!30}{75b10da55}

    \vspace{0.5cm}
    Using \inlinecode{\$ git status} told Git to compare the current state
    with the last known state directly from `origin/master'.
\end{frame}

\begin{frame}[t,fragile]\frametitle{Introducing Git}
    To see the last 2 commits we may do the following:

    \newbox{\abox}
    \begin{lrbox}{\abox}
    \begin{minipage}{\linewidth}
    \begin{lstlisting}[basicstyle=\footnotesize\ttfamily\color{white}]
$ git log -2
commit 75b10da55
Author: Pam <pam@usa.com>
Date:   Mon Mar 24 17:28:17 2014 -0500

    I have changed File a

commit 24b9da655
Author: David <david@milkandcheese.com>
Date:   Tue Mar 13 12:33:16 2014 -0500

    Included this month's cow deaths in File c
    \end{lstlisting}
    \end{minipage}
    \end{lrbox}
    \colorbox{black}{\usebox{\abox}}
\end{frame}

\begin{frame}[t]\frametitle{Introducing Git}
    To introduce her changes to the Central Server, Pam
    has to push her changes.

    \vspace{0.5cm}
    \inlinecode{\$ git push origin master}

    \vspace{0.5cm}
    The Central Server has been updated with Pam's changes.


\end{frame}