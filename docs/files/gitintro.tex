%!TEX root = ../main.tex

\begin{frame}[t]\frametitle{Introducing Git}
    Git is a distributional version control system most notably used
    by Github, the web-based hosting service for software development projects.

    We will use Git to demonstrate version control systems. For a basic
    introduction to git, we recommend \citet{ChaconHamano2009}, which
    can be accessed \href{http://git-scm.com/book}{here}. We heavily
    borrow heavily from this book here.
\end{frame}

\begin{frame}[t]\frametitle{Introducing Git}
    First imagine a git server, which contains all of the relevant
    files in a project.

    \begin{center}
        \tikzstyle{file}=[draw, rectangle, text centered, text width=2.5cm,
            fill=black!20, rounded corners]

        \begin{tikzpicture}

            \begin{scope}[local bounding box=bb]
                \node (server) {Central Server};
                \node [file, below of=server] (folder) {MyProject/};
                \node [file,below of=folder,xshift=2cm] (a) {File a};
                \node [file,below of=a] (b) {File b};
                \node [file,below of=b] (c) {File c};
                \node [file,below of=c,xshift=-2cm] (git) {.git/};
                \node [below of=git,xshift=0.3cm] (content) {...};

                \draw (folder) -- (0,-3);
                \draw (0,-3) -- (a.west);
                \draw (0,-3) -- (b.west);
                \draw (0,-3) -- (c.west);
                \draw (0,-3) -- (git.north);
                \draw (git.south) -- (content.west);
            \end{scope}

            \begin{pgfonlayer}{background}
                \node [fill=blue!20,fit=(bb)] {};
            \end{pgfonlayer}

            \draw [decorate,decoration={brace,amplitude=10pt}] (4,0) -- (4,-6) node [midway,xshift=1.5cm] (hash) {24b9da655};
            \node [right of=git,xshift=6cm,yshift=-1cm] (gitlab) {Git Storage};
            \node [below of=hash,xshift=1cm] (hashlab) {Commit Hash (version \#)};

            \draw [<-] (hash) -- (hashlab);
            \draw [<-] (git) -- (gitlab);
        \end{tikzpicture}
    \end{center}
\end{frame}

\begin{frame}[t]\frametitle{Introducing Git}
    When Pam clones this repository to her computer, she sees:
    \begin{center}
        \tikzstyle{file}=[draw, rectangle, text centered, text width=2.5cm,
            fill=black!20, rounded corners]

        \begin{tikzpicture}

            \begin{scope}[local bounding box=bb]
                \node (server) {Pam's Computer};
                \node [file, below of=server] (folder) {MyProject/};
                \node [file,below of=folder,xshift=2cm] (a) {File a};
                \node [file,below of=a] (b) {File b};
                \node [file,below of=b] (c) {File c};
                \node [file,below of=c,xshift=-2cm] (git) {.git/};
                \node [below of=git,xshift=0.3cm] (content) {...};

                \draw (folder) -- (0,-3);
                \draw (0,-3) -- (a.west);
                \draw (0,-3) -- (b.west);
                \draw (0,-3) -- (c.west);
                \draw (0,-3) -- (git.north);
                \draw (git.south) -- (content.west);
            \end{scope}

            \begin{pgfonlayer}{background}
                \node [fill=blue!20,fit=(bb)] {};
            \end{pgfonlayer}

            \draw [decorate,decoration={brace,amplitude=10pt}] (4,0) -- (4,-6) node [midway,xshift=1.5cm] (hash) {24b9da655};
            \node [right of=git,xshift=6cm,yshift=-1cm] (gitlab) {Git Storage};
            \node [below of=hash,xshift=1cm] (hashlab) {Commit Hash (version \#)};

            \draw [<-] (hash) -- (hashlab);
            \draw [<-] (git) -- (gitlab);
        \end{tikzpicture}
    \end{center}
\end{frame}

\begin{frame}\frametitle{Resources}
    Many resources available for git. 
    \href{http://stackoverflow.com/questions/tagged/git}{Stackoverflow} 
    will answer most questions. This 
    \href{http://stackoverflow.com/questions/315911/git-for-beginners-the-definitive-practical-guide}{post} is a great resource for beginners and advanced users alike.
\end{frame}